\documentclass[12pt,a4paper]{article}

\usepackage[polish]{babel}
\usepackage[utf8]{inputenc}
\usepackage{lmodern}
\usepackage{geometry}
\usepackage{hyperref}
\usepackage{listings}
\usepackage{xcolor}
\usepackage{graphicx}

\geometry{margin=2.5cm}

\lstset{
    basicstyle=\ttfamily\footnotesize,
    keywordstyle=\color{blue},
    commentstyle=\color{gray},
    stringstyle=\color{orange},
    breaklines=true,
    frame=single,
    tabsize=2,
    language=C
}

\title{Dokumentacja techniczna systemu pobierania danych pogodowych \\ \large Oracle + C\# Worker Service}
\author{CerbIT / System pogodowy}
\date{\today}

\begin{document}
\maketitle

\section{Wprowadzenie}

Celem projektu jest automatyczne pobieranie danych pogodowych z darmowego API
(Open-Meteo) i zapisywanie ich do bazy danych Oracle.
Proces działa jako \textbf{Windows Service} napisany w technologii .NET 8 (Worker Service),
uruchamiany cyklicznie co 10 minut.

Rozwiązanie składa się z następujących elementów:

\begin{itemize}
    \item baza danych Oracle – tabele \texttt{WEATHER\_LOCATIONS} oraz \texttt{WEATHER\_MEASUREMENTS},
    \item serwis Windows (Worker Service) uruchamiany w tle,
    \item komponent \texttt{WeatherSyncJob} odpowiedzialny za pobieranie i zapis danych,
    \item konfiguracja w pliku \texttt{appsettings.json}.
\end{itemize}

\section{Architektura systemu}

\subsection{Schemat logiczny}

\begin{verbatim}
+----------------------+      HTTP GET       +-----------------------+
| Windows Worker       | ------------------> | API Open-Meteo        |
| Service (.NET 8)     |                    +-----------------------+
|   - WeatherSyncJob   |
+----------+-----------+
           |
           | ODP.NET
           v
+------------------------------+
| Oracle Database             |
|   - WEATHER_LOCATIONS       |
|   - WEATHER_MEASUREMENTS    |
+------------------------------+
\end{verbatim}

\subsection{Opis komponentów}

\begin{itemize}
  \item \textbf{Worker (BackgroundService)}  
  Uruchamiany automatycznie przez system Windows. W pętli wykonuje zadania co
  10 minut.

  \item \textbf{WeatherSyncJob}  
  Główna logika biznesowa:
  \begin{itemize}
    \item test połączeń,
    \item pobranie danych pogodowych,
    \item serializacja/parsowanie JSON,
    \item zapis do Oracle.
  \end{itemize}

  \item \textbf{Oracle}  
  Baza przechowująca lokalizacje oraz pomiary pogody.
\end{itemize}

\section{Struktura bazy danych Oracle}

\subsection{Tabela LOCATION}

\begin{lstlisting}[language=SQL]
CREATE TABLE WEATHER_LOCATIONS (
    ID_LOCATION      NUMBER GENERATED BY DEFAULT AS IDENTITY PRIMARY KEY,
    COUNTRY_CODE     VARCHAR2(2) NOT NULL,
    CITY_NAME        VARCHAR2(100) NOT NULL,
    LATITUDE         NUMBER(9,6) NOT NULL,
    LONGITUDE        NUMBER(9,6) NOT NULL,
    ACTIVE_FLAG      CHAR(1) DEFAULT 'Y' NOT NULL
);
\end{lstlisting}

\subsection{Tabela MEASUREMENTS}

\begin{lstlisting}[language=SQL]
CREATE TABLE WEATHER_MEASUREMENTS (
    ID_MEASUREMENT   NUMBER GENERATED AS IDENTITY PRIMARY KEY,
    ID_LOCATION      NUMBER NOT NULL,
    MEASURED_AT      DATE NOT NULL,
    TEMP_C           NUMBER(5,2),
    IS_RAIN          CHAR(1),
    HUMIDITY         NUMBER(5,2),
    WIND_SPEED_MS    NUMBER(6,2),
    RAW_JSON         CLOB,
    INSERTED_AT      DATE DEFAULT SYSDATE NOT NULL,
    CONSTRAINT FK_LOC FOREIGN KEY (ID_LOCATION)
       REFERENCES WEATHER_LOCATIONS(ID_LOCATION)
);
\end{lstlisting}

\section{Konfiguracja systemu}

\subsection{Plik appsettings.json}

\begin{lstlisting}[language=json]
{
  "ConnectionStrings": {
    "Oracle": "User Id=USER;Password=PASS;Data Source=HOST:1521/SERVICE;"
  },
  "WeatherApi": {
    "BaseUrl": "https://api.open-meteo.com/v1/forecast",
    "Params": "hourly=temperature_2m,relativehumidity_2m,
               precipitation,wind_speed_10m&forecast_days=1&timezone=auto"
  }
}
\end{lstlisting}

\section{Implementacja serwisu Windows (.NET 8)}

\subsection{Plik Program.cs}

\begin{lstlisting}
var host = Host.CreateDefaultBuilder(args)
    .UseWindowsService()
    .ConfigureServices((context, services) =>
    {
        services.AddSingleton<WeatherSyncJob>();
        services.AddHostedService<Worker>();
    })
    .Build();

await host.RunAsync();
\end{lstlisting}

\subsection{Worker.cs – serwis cykliczny}

\begin{lstlisting}
public class Worker : BackgroundService
{
    private readonly WeatherSyncJob _job;

    public Worker(WeatherSyncJob job) => _job = job;

    protected override async Task ExecuteAsync(CancellationToken token)
    {
        while (!token.IsCancellationRequested)
        {
            await _job.RunOnce(token);
            await Task.Delay(TimeSpan.FromMinutes(10), token);
        }
    }
}
\end{lstlisting}

\section{Logika synchronizacji – WeatherSyncJob}

\subsection{Test połączeń}

\begin{lstlisting}
private async Task TestConnectionsAsync(CancellationToken token)
{
    using var conn = new OracleConnection(_conn);
    await conn.OpenAsync(token);
    
    var resp = await new HttpClient().GetAsync(_url, token);
}
\end{lstlisting}

\subsection{Parsowanie JSON}

\begin{lstlisting}
var doc = JsonDocument.Parse(json);
var hourly = doc.RootElement.GetProperty("hourly");
var last = hourly.GetProperty("time").GetArrayLength() - 1;

var measurement = new WeatherMeasurement {
    TempC = hourly.GetProperty("temperature_2m")[last].GetDouble(),
    ...
};
\end{lstlisting}

\subsection{Zapis pomiaru do Oracle}

\begin{lstlisting}
INSERT INTO WEATHER_MEASUREMENTS
(ID_LOCATION, MEASURED_AT, TEMP_C, IS_RAIN,
 HUMIDITY, WIND_SPEED_MS, RAW_JSON)
VALUES (:loc, :time, :temp, :rain, :hum, :wind, :json)
\end{lstlisting}

\section{Instalacja jako usługa Windows}

\subsection{Publikacja projektu}

\begin{lstlisting}
dotnet publish -c Release -r win-x64 --self-contained false -o publish
\end{lstlisting}

\subsection{Rejestracja usługi}

\begin{lstlisting}
sc create WeatherService binPath= "C:\Weather\WeatherService.exe"
sc start WeatherService
\end{lstlisting}

\section{Podsumowanie}

Przygotowany system umożliwia:

\begin{itemize}
  \item automatyczne i cykliczne pobieranie danych pogodowych,
  \item stabilne działanie jako usługa Windows,
  \item pełną integrację z Oracle (ODP.NET),
  \item możliwość łatwego rozszerzania o kolejne lokalizacje i parametry.
\end{itemize}

System jest gotowy do użycia produkcyjnego po dodaniu logowania (np. Serilog)
i mechanizmów retry/fallback przy błędach API.

\end{document}
